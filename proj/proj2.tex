% 0) Folhas de rosto (duas, sendo uma em português e outra em inglês) contendo título do projeto de pesquisa proposto, nome do Pesquisador Responsável (Supervisor) e do candidato à bolsa, Instituição Sede e resumo de 20 linhas.
 
% 1) Enunciado do problema: Qual será o problema tratado pelo projeto e qual sua importância? Qual será a contribuição para a área se bem sucedido? Cite trabalhos relevantes na área, conforme necessário.
%% ARS via visual analytics e considerando as propriedades das estruturas sociais (redes livres de escala e com comunidades e texto que depende da topologia)
%% Necessidade de visualizações simultâneas com layouts de grafos diferentes, plots e estatísicas
%% Métodos
%% Implementações
%% Análises derivads
%% Integração com dados ligados?
%% Citar literatura e software relacionados
 
% 2) Resultados esperados: O que será criado ou produzido como resultado do projeto proposto? Como os resultados serão disseminados?
%% Método e arcabouço para análise de redes sociais com ênfase em texto e topologia
%% Interface em código aberto para visualização interativa de redes sociais com relações topológicas e texto
%% Métodos para confirmação das hipóteses derivadas da visualização através de medidas
%% Análises de estruturas sociais do LOSD (Facebook, Twitter, IRC, ParticipaBR, AA)
%% Parcerias, e.g. com a Uni de Nebrasca
%% Contribuição para o proj FAPESP enviado pelo grupo
%% A solicitação deve definir a contribuição que o desenvolvimento do projeto proposto
%% e a formação prévia do candidato trarão ao grupo no qual se realizará o pós-doutoramento
 
% 3) Desafios científicos e tecnológicos e os meios e métodos para superá-los: explicite os desafios científicos e tecnológicos que o projeto se propõe a superar para atingir os objetivos. Descreva com que meios e métodos estes desafios poderão ser vencidos. Cite referências que ajudem os assessores que analisarão a proposta a entenderem que os desafios mencionados não foram ainda vencidos (ou ainda não foram vencidos de forma adequada) e que poderão ser vencidos com os métodos e meios da proposta em análise.
%% LOSD, temos dados a mais?
%% Layouts de grafos estabelecidos + Versinus
%% Projeções multidimensionais
%% Utilização de características esperadas dos dados para visualizações pertinentes (e.g. redes sociais são reportadas como livres de escala)
%% Técnicas de design (e.g. fontes, imagens conotativas/denotativas), psicologia (e.g. gestalt, kiki-bouba) e artes (e.g. teorias de cores, formas e estética)
%% Técnicas de visualização de dados/informação e de grafos/redes, visual analytics
%% HMI e/ou HCI?
%% Interface que comporte mais de uma visualização simultânea e interatividade
%% WebGL para possibilitar a renderização em tempo real de quantidades massivas de dados em navegadores usuais
%% Possivelmene a sonificação das estruturas encontradas nos dados para complementar interesse pelo usuário e o contato deste com os dados 
%% dados -> análise automática para definir uma visualização/interface inicial -> apresentação da visualização com widgets para controle pelo usuário -> possibilidade de mineração da interação e de anotação dos dados e da visualização global
%% Manutenção dos dados no sandbox local para reaproveitamento de estruturas elaboradas. Possibilidade de exportação/importação e comunicação direta entre clientes
 
% 4) Cronograma: Quando o projeto será completado? Quais os eventos marcantes que poderão ser usados para medir o progresso do projeto e quando estará completo? Caso o projeto proposto seja parte de outro projeto maior já em andamento, estime os prazos somente para o projeto proposto.
%% 24 meses
%% Leitura de artigos
%% Teste de software
%% Escrita de software
%% Escrita de artigos
%% Escrita/envio dos relatórios (apenas um ao final no dia 10 após a última bolsa?)
 
% 5) Disseminação e avaliação: Como os resultados do projeto deverão ser avaliados e como serão disseminados?
%% Publicações
%% SL
%% retorno das visualizações a análises para as comunidades analisadas
 
% 6) Outros apoios: Demonstre outros apoios ao projeto, se houver, em forma de fundos, bens ou serviços, mas sem incluir itens como uso de instalações da instituição que já estão disponíveis. Note que os autores das propostas selecionadas deverão apresentar carta oficial assinada pelo dirigente da instituição, comprometendo os recursos e bens adicionais descritos na proposta.
%% GSoC e Nebrasca
%% Proj temático
 
% 7) Bibliografia: liste as referências bibliográficas citadas nas seções anteriores.
