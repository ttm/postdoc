% Notas:
%% Talvez simplificar ao máximo e excluir itens para evitar descrétido dos revisores.
%% - Para processos que tramitam no SAGe, anexar o formulário à Solicitação de Mudança (SM) do tipo “Outra” que deve ser elaborada e submetida à FAPESP. ??? http://www.fapesp.br/bolsas/pd
%% espaço 1,5, Times New Roman 12 com margens de 3,5 cm à esquerda e 1,5 cm à direita
%% adaptar ao modelo proposto na página de PD. Resolver conflito entre o modelo desta página e na página de projeto temático http://www.fapesp.br/5389 . OK, feito. No caso de PT é feito o pedido pelo supervisor somente e depois feito um processo seletivo por ele.
%% ler http://www.fapesp.br/docs/formularios/arquivos/pibppd.docx como sugere a fapesp



% Intro
%% Visualização
%% Visualização de Grafos
%% Visualização de Redes Sociais
%% trabalho proposto
%% Vínculo ao projeto temáatico (talvez ligar na FAPESP para ver como melhor fazer isso com o proj em avaliação)
%% Sumário do projeto temático

%% Objetivos
% Contribuições para a área de visualização de redes e dados multidimensionais
% Obtenção de uma interface de visualização de redes sociais que considere topologia e texto
% Relatoria sobre os dados da LOSD
% Formalização das contribuições na forma de literatura científica

% (Plano de trabalho)

% Materiais
%% LOSD
%% Algo mais?

% Métodos
%% Layouts de grafos estabelecidos + Versinus
%% Projeções multidimensionais
%% Utilização de características esperadas dos dados para visualizações pertinentes (e.g. redes sociais são reportadas como livres de escala)
%% Técnicas de design (e.g. fontes, imagens conotativas/denotativas), psicologia (e.g. gestalt, kiki-bouba) e artes (e.g. teorias de cores, formas e estética)
%% Técnicas de visualização de dados/informação e de grafos/redes, visual analytics
%% HMI e/ou HCI?
%% Interface que comporte mais de uma visualização simultânea e interatividade
%% WebGL para possibilitar a renderização em tempo real de quantidades massivas de dados em navegadores usuais
%% Possivelmene a sonificação das estruturas encontradas nos dados para complementar interesse pelo usuário e o contato deste com os dados 
%% dados -> análise automática para definir uma visualização/interface inicial -> apresentação da visualização com widgets para controle pelo usuário -> possibilidade de mineração da interação e de anotação dos dados e da visualização global
%% Manutenção dos dados no sandbox local para reaproveitamento de estruturas elaboradas. Possibilidade de exportação/importação e comunicação direta entre clientes

% Resultados previstos
%% Método e arcabouço para análise de redes sociais
%% Interface para visualização interativa de redes sociais com relações topológicas e texto
%% Métodos para confirmação das hipóteses derivadas da visualização através de medidas
%% Análises de estruturas sociais do LOSD (Facebook, Twitter, IRC, ParticipaBR, AA)
%% Parcerias, e.g. com a Uni de Nebrasca
%% Contribuição para o proj FAPESP enviado pelo grupo
%% A solicitação deve definir a contribuição que o desenvolvimento do projeto proposto
%% e a formação prévia do candidato trarão ao grupo no qual se realizará o pós-doutoramento

% Cronograma
%% 24 meses
%% Leitura de artigos
%% Teste de software
%% Escrita de software
%% Escrita de artigos
%% Escrita/envio dos relatórios (apenas um ao final no dia 10 após a última bolsa?)

% Conclusões
