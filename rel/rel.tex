\documentclass[a4paper, 11pt]{article}
\usepackage{comment} % enables the use of multi-line comments (\ifx \fi) 
\usepackage{lipsum} %This package just generates Lorem Ipsum filler text. 
\usepackage{fullpage} % changes the margin
\usepackage[utf8]{inputenc}
\usepackage{color}
\usepackage[usenames,dvipsnames]{xcolor}
\usepackage{hyperref} 	%% Use to fix Figure or Table: ex: \begin{table}[H]
\hypersetup{
	%pagebackref=true,
	pdfcreator={LaTeX with abnTeX2},
	pdfkeywords={abnt}{latex}{abntex}{USPSC}{trabalho acadêmico}, 
	colorlinks=true,       		% false: boxed links; true: colored links
	linkcolor=blue,          	% color of internal links
	citecolor=blue,        		% color of links to bibliography
	filecolor=magenta,      		% color of file links
	urlcolor=blue,
	allbordercolors=black,
	bookmarksdepth=4
}

\begin{document}
%Header-Make sure you update this information!!!!
\noindent
\normalsize projeto: \textbf{2017/05838-3} \\
coordenadora: \textbf{Profa. Maria Cristina Ferreira de Oliveira} \\
bolsista: \textbf{Renato Fabbri} \\
Relatório I, bolsa TT5 \\

\section{Informação sobre o nível e período de usufruto da Bolsa}
A bolsa é TT5 (Treinamento Técnico nível 5).
O período é de 01/Set/2017 até hoje, dia 30/Jun/2018,
prazo final para entrega deste primeiro relatório.

% 
\section{Descrição das atividades do bolsista no projeto de pesquisa}
Neste período, pude aprofundar meus conhecimentos sobre visualização de dados
(em especial sobre \emph{analítica visual}),
compatibilizando meus conhecimentos e fluências antes adquiridas,
e adiantar os objetivos descritos no projeto inicial desta bolsa TT5 e no
projeto 2017/05838-3.
Mais especificamente:
\begin{itemize}
  \item Ajudei a ministrar, com a Profa. Cristina (coordenadora do projeto), as disciplinas SCC5836 e SCC0252, ambas a respeito de visualização de dados, na medida em que entendemos proveitoso para minha formação e andamento deste projeto.
  Em especial, elaborei trabalhos para os alunos entregarem e os corrigi, apresentei uma aula sobre Redes Complexas (com foco em redes sociais), e apresentei uma introdução à mineração de dados, e apresentei visualizações e algoritmos para a aplicações fundamentadas de PCA (\emph{principal component analysis}).
  \item Ministrei dois seminários: um sobre as implementações que fiz de \emph{layouts} de redes para o ccNetViz (usa WebGL para a visualização de estruturas grandes de dados); outro sobre cores.
  \item Publiquei 7 artigos no ENMC2017 (Encontro Nacional de Modelagem Computacional), três deles foram selecionados para publicação na Revista CEREUS (já submetidos).
  \item Disponibilizei no arXiv um artigo sobre o Vim e outro sobre Toki Pona, ambos tratando de realce de sintaxe (coloração de texto).
  \item Apreciei os livros principais da área de visualização de dados e alguns outros sugeridos pela Profa. Cristina. Em especial, li e fichei partes do livro da Munzner e do Ware.
  \item Desenvolvi ferramentas/plugins para Vim que auxiliam em nossas tarefas e estudos. Em especial, desenvolvi um plugin para lidar com cores (realce de sintáxe) e outro com bots (agentes conversacionais) desenvolvidos por mim mesmo. Este último para possibilitar a computação cognitiva, fundamental para a analítica visual.
  \item Concebemos, eu e a Profa. Cristina, o software a ser desenvolvido neste projeto. Desenvolvi até onde combinamos na última reunião (apenas uma janela com algumas \emph{widgets}). Tenho outros software desenvolvidos anteriormente que se encaixam nos moldes que concebemos.
  \item Organizei um evento de iniciação em Python no IFSC/USP, com João Bueno (Olist) e Ýdà Nã-Dãn (mitóloga e artista), de 12h no total.
  \item Escrevi um artigo a convite da Revisa V!RUS (Nomads/IAU/USP), acompanhado de um acervo de imagens de redes (disponibilizado como comunidade do Facebook e via link para imagens com a qualidade original).
  \item Concebemos alguns MOOCs (\emph{massive open online courses}), em especial um sobre ``pesquisa e desenvolvimento com tecnologias livres'', outro sobre ``música e áudio digital'', e um sobre ``visualização de dados'' (que prevemos calcar principalmente no livro da Munzner, notas de aulas, e artigos).
  \item Consideramos tanto a visualização quanto a sonificação de dados, o que nos levou ao conceito de ``Analítica Audiovisual''. Em especial, para isso, melhorei a referência principal que temos da descrição do áudio em termos de amostras LPCM (no arXiv).
  \item Melhorei a escrita, terminologia e pseudocódigos de um artigo escrito pelo Alan Valejo (doutorando) e os Profs. Cristina e Alneu de Andrade Lopes, sobre estratégias multinível para otimização e visualização. Está em minhas mãos terminar uma revisão minuciosa do artigo, para então receber melhoras e a chancela dos professores para submeter a alguma revista internacional. Não consta aqui caminho para este trabalho através dos links pois os colaboradores não estão trabalhando em aberto.
\end{itemize}

\section{Informar e justificar caso tenham ocorrido mudanças e, eventualmente, os ajustes realizados nas atividades de pesquisa do bolsista, em relação ao Plano de Atividades}
Demos bastante ênfase no meu aprofundamento na área e na compatibilização dos meus conhecimentos prévios com a área de Analítica Visual.
Por eu ter já bastante experiência com programação, confiamos que esta frente será encaminhada de Julho em diante.
Espero render mais, com mais foco (quase absoluto de agora em diante), nos objetivos descritos no projeto principal,
no projeto da bolsa TT5, e dados pela Profa. Cristina.


\section{Avaliação do impacto das atividades do bolsista sobre o andamento do projeto}
Apresentei ferramentas ao grupo de pesquisa, auxiliei na programação, escrita de artigos, em cursos regulares de visualização de dados,
enriqueci o contexto de pesquisa com técnicas e conteúdos novos.
Acredito estar auxiliando no andamento do projeto, em conformidade com a bolsa TT5.

\section{Juntar o histórico escolar atualizado do bolsista}
É o mesmo que o enviado ao início do projeto.
Já possuo doutorado, cursei mais matérias que o necessário tanto no mestrado quanto no doutorado,
consumo literatura científica e MOOCs regularmente, motivos pelos quais não consideramos a possibilidade
de eu assistir mais aulas.

\section{Se for o caso, especificar}
\subsection{O cronograma da próxima etapa do trabalho do bolsista no projeto}
Nos termos descritos acima, no projeto principal, e no projeto da bolsista TT5,
e contando com a direção da Profa. Cristina, não há necessidade de descrever novo calendário.
Apontamos nossa intenção de ministrar um curso sobre visualização de dados no semestre que vem.
Também de dar os próximos passos com maior acompanhamento mútuo dos interesses,
objetivos, conhecimentos, etc.

\subsection{Outras observações consideradas relevantes para a análise das atividades do bolsista por parte da FAPESP}
Pela limitação do tamanho deste relatório, prazo prorrogado já aprovado para o relatório da coordenadora,
e orientação da FAPESP via telefone, este relatório descreve apenas brevemente os feitos e o bolsista
auxiliará, na medida em que a coordenadora achar apropriado, a escrita do relatório do projeto principal
(a ser entregue em Julho).

\subsection{Links}
\begin{itemize}
  \item Artigos no arXiv: \url{https://arxiv.org/a/fabbri_r_1.html}
  \item Repositórios de código: \url{https://github.com/ttm/}
  \item Comunidade pública com direções para o artigo e galerias feitos a convite para a revista V!RUS: \url{https://www.facebook.com/groups/177946082897310/}
  \item Ementa do MOOC de pesquisa e desenvolvimento com tecnologias livres: \url{https://docs.google.com/document/d/1tPY1OedvmLzdIga0uK7DSH6j2UDcXEmZxOL_Kz2K3VI/edit#heading=h.duaka341x0pj}
  \item Ementa do MOOC de música e áudio computacionais: \url{https://docs.google.com/document/d/1O8y-W27P_ydeOXd97HBtLEeI0C0-IMhfq6AKWmqDW10/edit#heading=h.6jgd0ulugf3d}
  \item relato sobre o evento de iniciação em Python: \url{https://www.facebook.com/renato.fabbri/posts/10156562315734430}
\end{itemize}

\section{Apreciação do desempenho do bolsista}

Para ser escrito pela coordenadora.

\end{document}
% 
% 
% 
% 
% 
