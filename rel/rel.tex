\documentclass[a4paper, 11pt]{article}
\usepackage{comment} % enables the use of multi-line comments (\ifx \fi) 
\usepackage{lipsum} %This package just generates Lorem Ipsum filler text. 
\usepackage{fullpage} % changes the margin
\usepackage[utf8]{inputenc}

\begin{document}
%Header-Make sure you update this information!!!!
\noindent
\normalsize projeto: \textbf{2017/05838-3} \\
coordenadora: \textbf{Profa. Maria Cristina Ferreira de Oliveira} \\
bolsista: \textbf{Renato Fabbri} \\
Relatório I, bolsa TT5 \\

\section*{Informação sobre o nível e período de usufruto da Bolsa}
A bolsa é TT5 (Treinamento Técnico nível 5).
O período é de 01/Set/2017 até hoje, dia 30/Jun/2018,
prazo final para entrega deste primeiro relatório.

% 
\section*{Descrição das atividades do bolsista no projeto de pesquisa}
Neste período, pude aprofundar meus conhecimentos sobre visualização de dados
(em especial sobre \emph{analítica audiovisual}),
compatibilizando meus conhecimentos e fluências já antes adquiridas,
e adiantar os objetivos descritos no projeto inicial desta bolsa TT5 e no
projeto 2017/05838-3.
Mais especificamente:
\begin{itemize}
  \item Ajudei a ministrar, com a Profa. Cristina (coordenadora do projeto), as disciplinas SCC5836 e SCC0252, ambas a respeito de visualização de dados, na medida em que entendemos proveitoso para minha formação e andamento deste projeto.
  Em especial, elaborei trabalhos para os alunos entregarem e os corrigi, apresentei uma aula sobre Redes Complexas (com foco em redes sociais), e apresentei uma introdução à mineração de dados, e apresentei visualizações e algoritmos para a aplicações fundamentadas de PCA (\emph(principal component analysis).
  \item Dei dois seminários: um sobre as implementações que fiz de \emph{layouts} de redes para o ccNetViz (usa WebGL para a visualização de estruturas grandes de dados); outro sobre cores.
  \item Publiquei 7 artigos no ENMC (Encontro Nacional de Modelagem Computacional), três deles foram selecionados para publicação na Revista CEREUS (já submetidos).
  \item Disponibilizei no arXiv um artigo sobre o Vim e outro sobre Toki Pona, ambos tratando de realce de sintáxe (coloração de texto).
  \item Apreciei os livros principais da área de visualização de dados e alguns outros sugeridos pela Profa. Cristina. Em especial, li e fichei partes do livro da Munzner e do Ware.
  \item Desenvolvi ferramentas/plugins para Vim que auxiliam em nossas tarefas e estudos. Em especial, desenvolvi um plugin para lidar com cores (realce de sintáxe) e outro com bots (agentes conversacionais) desenvolvidos por mim mesmo. Este último como caminho para a computação cognitiva, fundamental para a analítica visual.
  \item Concebemos, eu e a Profa. Cristina, o software a ser desenvolvido neste projeto. Desenvolvi até onde combinamos na última reunião (apenas uma janela com algumas \emph{widgets}). Tenho outros software desenvolvidos anteriormente que se encaixam nos moldes que concebemos.
  \item Realizei um evento de iniciação em Python no IFSC/USP, com João Bueno (Olist) e Ýdà Nã-Dãn (mitóloga e artista).
  \item Escrevi um artigo a convite da Revisa V!RUS (Nomads/IAU/USP), acompanhado de um acervo de imagens de redes (disponibilizado como comunidade do Facebook e via link para imagens com qualidade original).
  \item Concebemos alguns MOOCs (\emph{massive open online courses}), em especial um sobre ``pesquisa e desenvolvimento com tecnologias livres'', outro sobre ``música e áudio digital'', e um sobre ``visualização de dados'' (que prevemos calcar principalmente no livro da Munzner, notas de aulas, e artigos).
  \item Consideramos tanto a visualização quanto a sonificação de dados, o que nos levou ao conceito de ``Analítica Audiovisual''. Em especial, para isso, melhorei a referência principal que temos da descrição do áudio em termos de amostras LPCM (no arXiv).
\end{itemize}

\section*{Informar e justificar caso tenham ocorrido mudanças e, eventualmente, os ajustes realizados nas atividades de pesquisa do bolsista, em relação ao Plano de Atividades}
\section*{Avaliação do impacto das atividades do bolsista sobre o andamento do projeto}
\section*{Juntar o histórico escolar atualizado do bolsista}
\section*{Se for o caso, especificar}
\subsection*{O cronograma da próxima etapa do trabalho do bolsista no projeto}
\subsection*{Outras observações consideradas relevantes para a análise das atividades do bolsista por parte da FAPESP}

\section*{Apreciação do desempenho do bolsista}

\end{document}
% 
% 
% 
% 
% 
