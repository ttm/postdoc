\documentclass[a4paper, 11pt]{article}
\usepackage{comment} % enables the use of multi-line comments (\ifx \fi) 
\usepackage{lipsum} %This package just generates Lorem Ipsum filler text. 
\usepackage{fullpage} % changes the margin
\usepackage[utf8]{inputenc}
\usepackage{color}
\usepackage[usenames,dvipsnames]{xcolor}
\usepackage{hyperref} 	%% Use to fix Figure or Table: ex: \begin{table}[H]
\hypersetup{
	%pagebackref=true,
	pdfcreator={LaTeX with abnTeX2},
	pdfkeywords={abnt}{latex}{abntex}{USPSC}{trabalho acadêmico}, 
	colorlinks=true,       		% false: boxed links; true: colored links
	linkcolor=blue,          	% color of internal links
	citecolor=blue,        		% color of links to bibliography
	filecolor=magenta,      		% color of file links
	urlcolor=blue,
	allbordercolors=black,
	bookmarksdepth=4
}
\usepackage[portuguese]{babel}

\begin{document}
%Header-Make sure you update this information!!!!
\noindent
\normalsize projeto: \textbf{2017/05838-3} \\
coordenadora: \textbf{Profa. Maria Cristina Ferreira de Oliveira} \\
bolsista: \textbf{Renato Fabbri} \\
Relatório I, bolsa TT5 \\

\section{Informação sobre o nível e período de usufruto da Bolsa}
A bolsa é TT5 (Treinamento Técnico nível 5).
O período é de 01/Set/2017 até hoje, dia 30/Jun/2018,
prazo final para entrega deste primeiro relatório.

\section{Descrição das atividades do bolsista no projeto de pesquisa}
Neste período, pude aprofundar meus conhecimentos sobre visualização de dados
(em especial sobre \emph{analítica visual}),
compatibilizando meus conhecimentos e fluências antes adquiridas,
e adiantar os objetivos descritos no projeto inicial desta bolsa TT5 e no
projeto 2017/05838-3.
Mais especificamente:
\begin{itemize}
  \item Participei, como ouvinte e colaborador, do oferecimento das disciplinas SCC5836 e SCC0252, ambas a respeito de visualização de dados, na medida em que entendemos proveitoso para minha formação e andamento deste projeto.
    Particularmente, além de estudar o livro texto, preparei e apresentei materiais sobre Redes Complexas (com foco em redes sociais), e apresentei uma introdução à mineração de dados, e apresentei visualizações e algoritmos para a aplicações fundamentadas de PCA (\emph{principal component analysis})~\cite{slidesDV}.
  \item Ministrei dois seminários: um sobre as implementações que fiz de \emph{layouts} de redes para o ccNetViz (usa WebGL para a visualização de estruturas grandes de dados)~\cite{sem1}; outro sobre cores~\cite{sem2}.
  \item Publiquei 7 artigos no ENMC2017 (Encontro Nacional de Modelagem Computacional)~\cite{e1,e2,e3,e4,e5,e6,e7}, três deles foram selecionados para publicação na Revista CEREUS (já submetidos)~\cite{e2,e3,e4}.
  \item Disponibilizei no arXiv um artigo sobre o Vim e outro sobre Toki Pona~\cite{arxiv}, ambos tratando de realce de sintaxe (coloração de texto).
  \item Apreciei os livros principais da área de visualização de dados e alguns outros sugeridos pela Profa. Cristina. Em especial, li e fichei partes do livro da Munzner~\cite{munzner} e do Ware~\cite{ware}.\cite{ward}
  \item Desenvolvi ferramentas/plugins para Vim que auxiliam em nossas tarefas e estudos. Em especial, desenvolvi um plugin para lidar com cores (realce de sintáxe) e outro com bots (agentes conversacionais) desenvolvidos por mim mesmo~\cite{prv}. Este último para possibilitar a computação cognitiva, fundamental para a analítica visual.
  \item Concebemos, eu e a Profa. Cristina, o software a ser desenvolvido neste projeto. Desenvolvi até onde combinamos na última reunião (apenas uma janela com algumas \emph{widgets}). Tenho outros software desenvolvidos anteriormente que se encaixam nos moldes que concebemos~\cite{repos}.
  \item Organizei um evento de iniciação em Python no IFSC/USP, com João Bueno (Olist) e Ýdà Nã-Dãn (mitóloga e artista), de 12h no total~\cite{bueno}.
  \item Escrevi um artigo a convite da Revisa V!RUS (Nomads/IAU/USP), acompanhado de um acervo de imagens de redes (disponibilizado como comunidade do Facebook e via link para imagens com a qualidade original)~\cite{virus}.
  \item Concebemos alguns MOOCs (\emph{massive open online courses}), em especial um sobre ``pesquisa e desenvolvimento com tecnologias livres''~\cite{tecl}, outro sobre ``música e áudio digital''~\cite{mus}, e um sobre ``visualização de dados'' (que prevemos calcar principalmente no livro da Munzner, notas de aulas, e artigos).
  \item Criei, geri e participei de comunidades e listas de email, e.g.~\cite{vimUse,fvim,fpln,fic,frc}
  \item Escrevi artigos completos na Wikipédia em português, e colaborei com artigos da Wikipédia em português e inglês. Em especial, escrevi estes artigos sobre inteligência computacional~\cite{wic,wsd,wib,wpe,wpo,wed,wsi,wcc,wce} e iniciei a edição de artigos sobre visualização de dados (e percepção) e.g.~\cite{wvd}. Há organizações conceituais diversas que estão prontas para serem passadas para a Wikipédia, e.g. sobre estatística e lógica difusa (\emph{fuzzy sets, logic, and systems}).
  \item Interagi com o professor Bento Dias da Silva (FCLAR/UNESP), por onde pude discutir computação cognitiva com foco na linguística. Disponibilizamos a wordnet de Verbos em Português, em todos os formatos úteis que ele tem~\cite{bento}.
  \item Consideramos tanto a visualização quanto a sonificação de dados, o que nos levou ao conceito de ``Analítica Audiovisual''. Em especial, para isso, melhorei a referência principal que temos da descrição do som e da música em termos de amostras PCM~\cite{arxiv}.
  \item Melhorei a escrita, terminologia e pseudocódigos de um artigo escrito pelo Alan Valejo (doutorando) e os Profs. Cristina e Alneu de Andrade Lopes, sobre estratégias multinível para otimização e visualização. Está em minhas mãos terminar uma revisão minuciosa do artigo, para então receber melhoras e a chancela dos professores para submeter a alguma revista internacional. Não consta aqui caminho para este trabalho pois os colaboradores não estão trabalhando em aberto.
\end{itemize}

\section{Informar e justificar caso tenham ocorrido mudanças e, eventualmente, os ajustes realizados nas atividades de pesquisa do bolsista, em relação ao Plano de Atividades}
Demos bastante ênfase no meu aprofundamento na área e na compatibilização dos meus conhecimentos prévios com a área de Analítica Visual.
Por eu ter já bastante experiência com programação, confiamos que esta frente será encaminhada de Julho em diante.
Espero render mais, com mais foco (quase absoluto de agora em diante), nos objetivos descritos no projeto principal,
no projeto da bolsa TT5, e dados pela Profa. Cristina.

\section{Avaliação do impacto das atividades do bolsista sobre o andamento do projeto}
Apresentei ferramentas ao grupo de pesquisa, auxiliei na programação, escrita de artigos, em cursos regulares de visualização de dados,
enriqueci o contexto de pesquisa com técnicas e conteúdos novos.
Acredito estar auxiliando no andamento do projeto, em conformidade com a bolsa TT5.

\section{Juntar o histórico escolar atualizado do bolsista}
Não se aplica.

\section{Apreciação do desempenho do bolsista (escrito pela coordenadora)}
O bolsista tem formação e capacidade para contribuir para o projeto, e tem estudado a literatura da área. Minha 
avaliação é que tem faltado foco e o estabelecimento de prioridades nas ações, o que já discutimos. Minha 
expectativa é que ele direcione seus esforços na execução das tarefas que definimos como relevantes e  necessárias 
para o andamento do projeto.

\begin{thebibliography}{99}
\bibitem{slidesDV}
  Slides para colaboração com disciplinas de Visualização de Dados \url{https://github.com/ttm/dataVisualization/tree/master/slides}

\bibitem{sem1}
  Slides do colóquio sobre Layouts de Redes em Javascript, WebGL e ccNetViz \url{https://github.com/ttm/prv/blob/master/pack/prv/opt/wiki/aux/wiki/legacy/slides01MarLayoutsJS.md}

\bibitem{sem2}
  Slides do colóquio sobre Cores na Visualização de dados \url{https://github.com/ttm/prv/blob/master/pack/prv/opt/wiki/aux/wiki/dataav/slides2018May11}, \url{https://github.com/ttm/prv/blob/master/pack/prv/opt/wiki/aux/wiki/dataav/slides2018May18}

\bibitem{e1}
	Fabbri, R. (2017). Enhancements of linked data expressiveness for ontologies.
  Encontro Nacional de Modelagem Computacional 2017 (XX ENMC).  From \url{https://github.com/ttm/ontologyEnhancements/raw/master/article.pdf}

\bibitem{e2}
	Fabbri, R., \& de Oliveira, M. C. F. (2017). Audiovisual Analytics Vocabulary and Ontology (AAVO): initial core and example expansion. arXiv preprint arXiv:1710.09954.

\bibitem{e3}
        Fabbri, R., \& Pisani, M. M. (2017). Egalitarian aspects of scale-free networks. arXiv preprint arXiv:1711.06199.

\bibitem{e4}
  Fabbri, R., \& Borges, F. M. (2017). Text Mining Descriptions Of Dreams: aesthetic and clinical efforts. arXiv preprint arXiv:1711.04609.

\bibitem{e5}
  Fabbri, R. (2017). The Algorithmic-Autoregulation (AA) Methodology and Software: a collective focus on self-transparency. arXiv preprint arXiv:1711.04612.
  
\bibitem{e6}
  Fabbri, R., \& De León, F. G. (2017). A Statistical Distance Derived From The Kolmogorov-Smirnov Test: specification, reference measures (benchmarks) and example uses. arXiv preprint arXiv:1711.00761.

\bibitem{e7}
  Fabbri, R., \& Garcia, L. H. (2017). A Simple Text Analytics Model To Assist Literary Criticism: comparative approach and example on James Joyce against Shakespeare and the Bible. arXiv preprint arXiv:1710.09233.

\bibitem{arxiv}
  Fabbri, R. et al. Artigos no arXiv: \url{https://arxiv.org/a/fabbri_r_1.html}

\bibitem{munzner}
	Munzner, T. (2014). Visualization analysis and design. CRC press.

\bibitem{ware}
  Ware, C. (2012). Information visualization: perception for design. Elsevier.

\bibitem{ward}
	Ward, M. O., Grinstein, G., \& Keim, D. (2010). Interactive data visualization: foundations, techniques, and applications. CRC Press.

\bibitem{prv}
  Fabbri, R. Python, RDF, and Vim: syntax highlighting, bots, AA, wiki, and more: \url{https://github.com/ttm/prv/}

\bibitem{repos}
  Fabbri, r. Repositórios de código: \url{https://github.com/ttm/}

\bibitem{bueno}
  Relato sobre o evento de iniciação em Python: \url{https://www.facebook.com/renato.fabbri/posts/10156562315734430}

\bibitem{virus}
  Fabbri, R. Comunidade pública com direções para o artigo e galerias feitos a convite para a revista V!RUS.  \url{https://www.facebook.com/groups/177946082897310/}

\bibitem{tecl}
  Ementa do MOOC de pesquisa e desenvolvimento com tecnologias livres: \url{https://docs.google.com/document/d/1tPY1OedvmLzdIga0uK7DSH6j2UDcXEmZxOL_Kz2K3VI/edit#heading=h.duaka341x0pj}

\bibitem{mus}
  Ementa do MOOC de música e áudio computacionais: \url{https://docs.google.com/document/d/1O8y-W27P_ydeOXd97HBtLEeI0C0-IMhfq6AKWmqDW10/edit#heading=h.6jgd0ulugf3d}

\bibitem{vimUse}
  Vim users email list: \url{https://groups.google.com/forum/#!forum/vim_use}

\bibitem{fvim}
  Vim users facebook group: \url{https://www.facebook.com/groups/124928894848184/}

\bibitem{fpln}
  NLP, Text Mining and Computational Linguistics facebook group: \url{https://www.facebook.com/groups/plnbr/}

\bibitem{fic}
  Computational Intelligence facebook group: \url{https://www.facebook.com/groups/351214512041180/}

\bibitem{frc}
  Complex Systems and Networks facebook group: \url{https://www.facebook.com/groups/1522716907806470}

\bibitem{wic}
  Fabbri, R. Inteligência Computacional. Wikipédia (artigo reescrito e complementado): \url{https://pt.wikipedia.org/wiki/Intelig%C3%AAncia_computacional}

\bibitem{wsd}
  Fabbri, R. Sistema Difuso de Controle. Wikipédia (artigo criado e escrito): \url{https://pt.wikipedia.org/wiki/Sistema_difuso_de_controle}

\bibitem{wib}
  Fabbri, R. Inferência Bayesiana. Wikipédia (artigo reescrito e complementado): \url{https://pt.wikipedia.org/wiki/Infer%C3%AAncia_bayesiana}

\bibitem{wpe}
  Fabbri, R. Programação Estruturada. Wikipédia (artigo reescrito e complementado): \url{https://pt.wikipedia.org/wiki/Programa%C3%A7%C3%A3o_estruturada}

\bibitem{wpo}
  Fabbri, R. Programação Orientada ao Objeto. Wikipédia (artigo reescrito e complementado): \url{https://pt.wikipedia.org/wiki/POO}

\bibitem{wed}
  Fabbri, R. Estruturas de Dados. Wikipédia (artigo reescrito e complementado): \url{https://pt.wikipedia.org/wiki/Estrutura_de_dados}

\bibitem{wsi}
  Fabbri, R. Inteligência de Enxame. Wikipédia (artigo reescrito e complementado): \url{https://pt.wikipedia.org/wiki/Intelig%C3%AAncia_de_enxame}

\bibitem{wcc}
  Fabbri, R. Computação Cognitiva. Wikipédia (artigo criado e escrito): \url{https://pt.wikipedia.org/wiki/Computa%C3%A7%C3%A3o_cognitiva}

\bibitem{wce}
  Fabbri, R. Computação Evolucionária. Wikipédia (artigo reescrito e complementado): \url{https://pt.wikipedia.org/wiki/Computa%C3%A7%C3%A3o_evolucion%C3%A1ria}

\bibitem{wvd}
  Fabbri, R et al. Visualização de Dados. Wikipédia (artigo sendo reescrito e complementado): \url{https://pt.wikipedia.org/wiki/Visualiza%C3%A7%C3%A3o_de_dados}

\bibitem{bento}
  Silva, B. D. da. Wordnet de verbos em português em alguns formatos. Github: \url{https://github.com/ttm/wordnets}

\end{thebibliography}


\end{document}
% 
% 
% 
% 
% 
