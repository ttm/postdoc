\documentclass[a4paper,12pt]{article}
\usepackage{titling}
\usepackage[brazil]{babel}
\usepackage[latin1]{inputenc}
\usepackage{graphicx}
\usepackage[dvipsnames]{xcolor}
\usepackage{hyperref}
\usepackage{setspace}
\onehalfspacing
\usepackage{geometry}
\pagenumbering{gobble}
\geometry{
	a4paper,
        left=35mm,
        top=15mm,
}
% \linespread{1.25}
% \hypersetup{linkbordercolor=blue}
\hypersetup{colorlinks=true,
linkcolor=[rgb]{0.2,0.5,0.2},
citecolor=magenta}
\bibliographystyle {ieeetr}
\title{Anal�tica Visual (Visual Analytics) para processos de difusao em redes complexas}
\author{\textbf{Proponente:} Dr. Renato Fabbri\\
\textbf{Supervisora:} Profa.  Dra. Maria Cristina Ferreira de Oliveira\\
Instituto de Ci�ncias Matem�ticas e de Computa��o,\\
Universidade de S�o Paulo (ICMC-USP)}
\date{\today}
\usepackage{blindtext}

\begin{document}

\maketitle

\noindent
\textbf{Resumo:}
Os processos de difusao sao modelados de diversas formas, e o estudo
destes processos em redes eh uma area ampla.
Englobam e.g. modelos inspirados em e aplicacoes para epidemiologia, redes de relacoes entre genes e proteinas, de informacao em estruturas sociais humanas, de estado fisico ou de sistemas computacionais.
Este projeto prop�e o desenvolvimento de m�todos e software de anal�tica visual (\emph{visual analytics}) para processos de difusao em redes complexas.
Em especial, ha propostas de modelos potencialmente ineditos que podem ser estudados com o auxilio de simulacoes visuais, mesmo nos casos em que os metodos numericos e a analitica algebrica e torna-se custosos, dificeis ou de exposicao laboriosa.
O proponente deste plano de trabalho colabora com o VICG/ICMC/USP,
desde que iniciou o pos-doutorado adiante adaptado para um Treinamento Tecnico V
(FAPESP XXXX), e pesquisou a caracterizacao de redes complexas sociais no seu doutorado.
Assim, ha artigos e ferramentas desenvolvidas atraves dos quais
o pesquisador retomarah a pesquisa e estabelecerah coesao dentre as pesquisas
realizadas e propostas.
Por exemplo, foram concebidos metodos e ferramntas de visualizacao de redes
bipartidas e de redes logitudinais 
(i.e. evolutivas, dinamicas;
que ganham e perdem vertices e arestas em uma sucessao de eventos).
Destacam-se os metodos e ferramentas para visualizacao de redes bipartidas
assistidos por estrategias multinivel.
Ao incrementar as contribuicoes jah efetivadas,
algumas interfaces e metodos em software, e artigos,
estarao em melhor estado para continuidade e colaboracao
em outras parcerias e orientacoes academicas.


\noindent\\
\textbf{Palavras-chave:} Analitica visual, Redes sociais, Redes complexas, Minera��o de texto.

\title{Visual Analytics of text and topology in social networks}
\author{\textbf{Proponent:} Dr. Renato Fabbri\\
\textbf{Supervisor:} Prof.  Dr. Maria Cristina Ferreira de Oliveira\\
Institute of Mathematics and Computational Sciences,\\
University of S�o Paulo (ICMC-USP)}


\maketitle

\noindent
\textbf{Abstract:}
\noindent\\
\textbf{Keywords:} Visual analytics

\newpage
\clearpage
\pagenumbering{arabic}
\setcounter{page}{1}
% formula��o do problema, objetivo, justificativa, metodologia e cronograma de execu��o
\section{Formulacao do problema}
"Essentially, all models are wrong, but some are useful" (George Box, 1987)

\subsection{Aspectos eticos}
Nos trabalhos~\cite{} estao consideradas questoes eticas da pesquisa cientifica
envolvendo seres humanos.
Em especial, foram desenvolvidos emprestimos da antropologia para a pesquisa em fisica,
sob o acompanhamento de academicos mais experientes, principalmente e nominalmente:
Profa. Dra. Marilia Pisani (Filosofia, UFABC), Profa. Dra. Deborah Antunes (Psicologia, UFC), e Prof. Dr. Massimo Canevacci (Sapienza Universita di Roma).
Em resumo, os aspectos eticos foram amenizados atraves do estudo das estruturas sociais
do proprio pesquisador (emprestimo da ``escrita de diarios'', tecnica da antropologia etnografica),
e da manutencao da transparencia da pesquisa com manutencao para acesso publico
de textos, dados e software (emprestimo da cultura livre: contribuicao para o legado
publico da humanidade de conhecimentos e tecnologias).
Com isso, foi possivel realizar alguns experimentos nas redes do proprio pesquisador,
e a consideracao destes experimentos em documentos cientificos,
sem o investimento de tempo para vencer burocracias e procedimentos de comites de etica.

No ambito deste plano de trabalho, o pesquisador concentrara esforcos no desenvolvimento
de modelos, ferramntas em software, simulacoes, e relatoria cientifica.
Este aspecto da pesquisa nao implica na necessidade da aprovacao de qualquer
comite de etica.
Mesmo assim, caso haja espaco e pertinencia para experimentos em estruturas sociais,
as propostas dos experimentos serao consideradas pelos pesquisadores
proponente e pela responsavel para submissao aos comites de etica cabiveis.

\subsection{Sincronizacao}
Simplificadamente , a sincronizacao eh tratada em conjunto com ou como um efeito permanente na rede~\cite{sync-consensus}.
Um pouco por reuso do vocabulario e pelos modelos de interesse,
chamamos aqui de sincronizacao uma classe de processos e modelos de difusao.

Alguns modelos desta classe foram criados para desenvolvimento tecnologico pessoal do pesquisador, e visa a propagacao de informacao e concordancias sociais.
Considere um grafo (i.e. uma rede) $G=(V,E)$, em que $V=\{v_i\}_0^{N-1}$ sao vertices, e $E=\{e_i = (v_j, v_k)\}_{i=0}^{z-1}$ sao arestas.
Os pressupostos gerais sao:
\begin{itemize}
  \item alguns vertices $v_i$, $|v_i| = a << N$, sao ativados para iniciar a propagacao, chamados sementes ou vertices iniciais.
  \item Cada vertice, ao ser ativado, ativa $b$ (constante) vizinhos e nao pode mais ser ativado. Caso o vertice nao possua $b$ vizinhos nao-ativados, ativara todos os vizinhos possivel.
\end{itemize}

Os criterios de escolha das sementes, $a$, $b$, sao arbitrarios.
Alem disso, ha variantes, e.g. os vertices ativados podem ser novamente ativados,
ou $b$ pode ser variante no tempo, randomico ou dependente de caracteristicas dos
vertices (e.g. grau).
Pode-se tambem impor restricoes adicionais, dentre os quais a mais comum eh que todos os vertices sejam ativados (sincronizacao completa).

As experiencias descritas em~\cite{virus,dissertacao,antphys1,ap2,ap3}
resultaram na imposicao de pouca atividade de cada vertice,
e propagacao desta atividade,
pois demoraram meses alguns procedimentos (e.g. os realizados em Dez/2012 ateh Mar/2013).
Os experimentos mais efemeros eram baseados em comunicacao (i.e. ativacao) de alguns poucos vertices e nao trouxeram mudanca nitida na estrutura e funcionamento da rede (assuntos, trocas, etc).
Tambem foi obtida a heuristica de comecar pelos vertices menos conectados, os perifericos~\cite{stab,dissertacao}.
O detalhamento das justificativas e evidencias que corroboram estas
caracteristicas foge ao escopo deste documento e encontra-se na literatura citada.

Por fim, pode-se utilizar tecnicas avancadas de ciencia de redes.
Considere o modelo basico resultante dos pressupostos gerais e consideracoes acima.
Ou seja, $a$ sementes, $b$ ativacoes por ativacao, 

Para exemplificar um processo de sincronizacao, considere
uma estrategia multinivel, em uma rede original $G_0$ eh representada
como uma sucessao de redes $M = \{G_i\}_0^{m-1}$ tal que se
$i < j$, $N_i \leq N_j$ e $z_i \leq z_j$.
Para a obtencao de $G_j$ a partir de $G_i$, sao determinados
os conjuntos de vertices que serao colapsados em supervertices
atraves de algum dentre os varios algoritmos de \emph{matching}.
$V_j$ resulta dos supervertices e dos vertices que nao fizeram parte
de nenhum conjunto colapsado.
$E_j$ resulta das arestas implicadas pela rede $G_i$ e pela correspondencia
entre $V_i$ e $V_j$, um algoritmo que depende das caracteristicas da rede
(simples, com peso, direcionada, bipartida, multicamada, heterogenea, etc).

Com o objetivo de obter sementes iniciais e a sequencia de vizinhos a serem ativados,

Um conjunto de vertices para colapso consistirah na escolha dos $b$ vizinhos mais conectados de um vertice.
Este vertice serah o vertice mais conectado que ainda nao pertence a um destes conjuntos.
O processo continuarah ate na rede menor restem somente $b$ vertices que 
nao participaram de nenhum colapso.

A introducao de ruido na escolha dos vertices e vizinhos a serem colapsados
permite a comparacao e escolha entre as diferentes colecoes de sementes $S=\{v_i\}_0^{s - 1}$ e
arestas $A$ relacionadas a cada vertice $A=\{(v_i, \{v_j\})_0^{a_i - 1}\}_0^{a-1}$ a serem exercitadas na propagacao da ativacao.

Na propagacao de um patogeno, o contagio eh muitas vezes modelado sendo igualmente
possivel por cada aresta.
Este eh realmente o caso quando o patogeno eh assintomatico.
Ja em quadros mais complicados e obitos, o individuo pode contagiar diretamente apenas
um numero limitado de pessoas, pois retira-se para tratamento ou internacao.
Alem disso, dado que a rede observada eh integrada a outras redes, 
o patogeno eh introduzido na rede $G$ por redes nao observadas, sem a necessidade
de arestas em $E$.

Estas ultimas consideracoes sao uteis para estudos epidemiologicos (e.g. SARs-Covid-19).
O modelo em que $A$ eh determinado adequa-se melhor para aplicaoes comerciais (e.g. Marketing Multinivel)
e para \emph{crowdsourcing} de infomacao/dados (e.g. democracia liquida e formacao de redes).

\subsection{Trabalhos relacionados de outros autores}\label{subsec:tr}
Os processos de difusao em redes complexas tem recebido crescente atencao
na literatura cientifica.
De fato, os modelos sao utilizados para uma classe vasta de fenomenos que ganharam
mais relevancia na ciencia recente, como redes de interacoes entre proteinas e genes
em celulas, de contagio na epidemiologia, de informacao, opiniao e fofoca em plataformas e estruturas sociais.

Eh possivel, neste contexto, discerninr duas abordagens paradigmaticas.
Ha a abordagem abstrata, em que os vertices e arestas nao sao definidos
para alem de suas definicoes matematicas ou em que sao minimizadas as particularidades
dos sistemas de interesse.
Ha abordagens bastante especializadas para os sistemas de interesse (e.g. biologico, social, fisico, tecnologico), e nesta linha provavelmente se destacam as ``analises de difusao baseadas em redes'' e as redes medicas.

Interessa-nos em especial adaptar os modelos para utilizacao de estrategias multinivel,
e redes em que ha uma modificacao sequencial nos vertices e arestas considerados.
A caracterizacao de novos modelos, como o descrito na secao anterior, seguirah a pertinencia como reforcada pelos trabalhos dos pesquisadores envolvidos ou por lacunas
encontradas na literatura.

Como esperado,
o propontente acompanha a pesquisa em andamento na ciencia da complexidade, redes complexas, e redes de difusao.
Este acompanhamento inclui os artigos e livros da bibliografia, e fontes que se mostraram uteis
no decorrer da pos-graduacao e projeto FAPESP ().
Tais fontes incluem MOOCs\footnote{a}, software\footnote{graphology, networkx}, assinatura e contribuicoes em paginas da Wikipedia,
e notas publicadas por instituicoes competentes\footnote{\url{https://www.santafe.edu/research/projects/transmission-sfi-insights-covid-19}.}

\section{Objetivo}
O objetivo deste plano de trabalho eh o desenvolvimento de modelos de processos de difusao em redes, e a implementacao de interfaces de visualizacao

\subsection{Objetivos especificos}
\section{Justificativa}

\subsection{Historico de pesquisa especializada do proponente}

\section{Metodologia}
O trabalho sera acompanhado de constante aprofundamento teorico na area.
Com base na experiencia do pesquisador na area, ha pertinencia na visita a
MOOCs\footnote{Sigla de \emph{Massive Online Open Courses}},
e.g. \emph{Network Dynamics of Social Behavior} (\url{https://www.coursera.org/learn/networkdynamics/home/welcome}), 
e este segmento do \emph{Curso do cara da stanford} (\url{https://saoid}).

Serao garantidos um minimo de 12 horas de dedicacao semanais com foco absoluto neste plano de trabalho,
a serem distribuidos entre aprofundamentos teoricos,
desenvolvimento de metodos e modelos,
desenvolvimento e manutencao de software,
articulacao com outros pesquisadores e instituicoes,
e escrita de artigos e relatorios cientificos.
Este eh o numero minimo de horas semanais previstas na Resolu��o CoPq N� 7413.
O pesquisador estarah mantendo outras atividades de pesquisa e engenharia de software
relacionadas aa ciencia de redes, habito mantido ha mais de 10 anos.
Caso novas tecnologias e metodos
desenvolvidos sejam diretamente relacionados a este plano de trabalho,
o pesquisador entrarah em contato com a Pro-Reitoria de Inovacao,
como previsto na resolucao supracitada.


\section{Cronograma de execucao}
\begin{enumerate}
  \item Concepcao do plano de trabalho e estabelecimento da parceria.
  \item Aprofundamento do conhecimento de teoria de redes complexas e processos de difusao em redes, em conformidade com a Secao~\label{subsec:tr}.
  \item Acr�scimos aos modelos atuais de anal�tica visual e visualiza��o de dados aplicados � an�lise de redes sociais,
  com o foco no participante da rede, nos pesquisadores em potencial
  e na classifica��o/tipologia de redes e participantes.
  \item Implementa��o computacional.
  Estamos j� implementando layouts para grafos no ccNetViz.
  \item Escrita e publica��o dos resultados em artigos.
  Esta etapa est� j� em andamento pois possu�mos diversos escritos com resultados relacionados
  � minera��o e visualiza��o de dados de redes sociais que est�o sendo submetidos para publica��o.
  \item Trocas com pesquisadores externos, estabelecimento de colabora��es.
  \item Elabora��o do relat�rio cient�fico final.
\end{enumerate}

\begin{table}[h!]
\begin{center}
  \begin{tabular}{  c |   c |   c  c    | c    }
      & \multicolumn{1}{|c|}{2017} & \multicolumn{2}{|c|}{2018} & \multicolumn{1}{|c}{2019} \\
    Atividade & 2$^{\circ}$ & 1$^{\circ}$ & 2$^{\circ}$ & 1$^{\circ}$ \\ \hline 

    1 & $\bullet$ & $\bullet$ & & \\
    2 & $\bullet$ & $\bullet$ &  &  \\
    3 & $\bullet$ & $\bullet$ & $\bullet$ & $\bullet$  \\
    4 & $\bullet$ & $\bullet$ & $\bullet$ & $\bullet$  \\
    5 & $\bullet$ & $\bullet$ & $\bullet$ & $\bullet$  \\
    6 & & & & $\bullet$  \\
  \end{tabular}
\caption{Cronograma de atividades ao longo dos semestres.}
\label{tab:cron}
\end{center}
\end{table}
\bibliography{references}
\end{document}
